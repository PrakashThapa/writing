
\subsubsubsection{XMARK in MongoDB}
	MongoDB's collections have  similar and related documents together that helps better indexing ultimately improve in performance. It will not worth to have single collection of whole XMark data as mentioned in \ref{xmark-nosql}. As we shown in Fig.~\ref{fig:xmark-schema}. we create each collection of each substructure in such a way that we will not lose data as well as most representation of xmark data. For Mongodb, the data model of in \ref{xmark-nosql} slightly change. Each \textit{type} represented as a collection. So we will have six collections and type is already represented by the collections. we don't need key/value of type in our document. For \textit{items},  \textit{regions} contains the name of region of each item as usual. 
	
	Finally as we mentioned in Indexing, the \textit{id} attribute of each these documents will be renamed to \textit{\_id} for default indexing. In case of \texttt{closed\_auctions} and \texttt{catgraph}, system will automatically generate \texttt{\_id} which is useless for our application.
\label{code:mongodb-person0}
\begin{fakeJSON}[label=json123,caption=Mongodb data representation of XMARk data]
{
	"_id": "person0",
	"name": "Kasidit Treweek",
	"emailaddress": "mailto:Treweek@cohera.com",
	"phone": "+0 (645) 43954155",
	"homepage": "http://www.cohera.com/~Treweek",
	"creditcard": "9941 9701 2489 4716",
	"profile": {
		"income": 20186.59,
		"interest": [{
			"category": "category251"
		}],
		"education": "Graduate School",
		"business": "No"
	}
}
\end{fakeJSON} 
	
\subsubsubsection{Queries}
	coming soon ...
	
	
