\documentclass{thesis}

\usepackage[nottoc]{tocbibind}

% hier namen etc. einsetzen
\fullname{Falco Nogatz}
\email{falco.nogatz@uni-ulm.de}
\headline{From XML Schema to JSON Schema -\\Comparison and Translation\\with Constraint Handling Rules}
\titel{Thema}
\jahr{2013}
\matnr{718320}
\gutachterA{Prof. Dr. Thom Frühwirth}
%\gutachterB{Gutachter 2}
%\betreuer{Betreuer}
\typ{Bachelor Thesis }
\fakultaet{Engineering and\\Computer Science}
\institut{Institute of Software\\Engineering and Compiler\\Construction}

% Falls keine Lizenz gewünscht wird bitte den folgenden Text entfernen.
% Die Lizenz erlaubt es zu nichtkommerziellen Zwecken die Arbeit zu
% vervielfaeltigen und Kopien zu machen. Dabei muss aber immer der Autor
% angegeben werden. Eine kommerzielle Verwertung ist für den Autor
% weiter möglich.
\license{
This work is licensed under the Creative Commons.
Attribution-NonCommercial-ShareAlike 3.0 License. To view a copy of this
license, visit http://creativecommons.org/licenses/by-nc-sa/3.0/ or send a
letter to Creative Commons, 543 Howard Street, 5th Floor, San Francisco,
California, 94105, USA. \\ Satz: PDF-\LaTeXe
}

\hypersetup{%
	pdftitle=\pdfescapestring{\thetitel},
	pdfauthor={\thefullname},
 	pdfsubject={\thetyp},
}


% trennungsregeln
\hyphenation{Sil-ben-trenn-ung}


\lstdefinelanguage{JSON}{
  keywords={typeof, new, true, false, catch, function, return, null, catch, switch, var, if, in, while, do, else, case, break},
  ndkeywords={class, export, boolean, throw, implements, import, this},
  sensitive=false,
  comment=[l]{//},
  morecomment=[s]{/*}{*/},
  morestring=[b]',
  morestring=[b]"
}

\lstdefinelanguage{Prolog}{
  keywords={typeof, new, true, false, catch, function, return, null, catch, switch, var, if, in, while, do, else, case, break},
  ndkeywords={class, export, boolean, throw, implements, import, this},
  sensitive=false,
  comment=[l]{//},
  morecomment=[s]{/*}{*/},
  morestring=[b]',
  morestring=[b]"
}


\usepackage{tikz}
\usetikzlibrary{trees}


\begin{document}
\frontmatter
\maketitle
% impressum
\clearpage
\impressum

\cleardoublepage
% ab hier zeilenabstand 1,4 fach 10pt/14pt
\setstretch{1.4}

XML and NoSQL database are are two growing field in second generation database system, They share some similarities as well as they have some significant difference. 
This thesis focus on the  comparative analysis of these two database system. Based on the Use cases and existing solution, we will discuss the data processing, query pattern and  data retrial from these

\cleardoublepage

%\input{acknowledgement}

% inhaltsverzeichnis einfügen
\tableofcontents

\mainmatter

\tikzstyle{every node}=[draw=black,thick,anchor=west,font=\small]
\tikzstyle{json}=[draw=green,fill=green!30]
\tikzstyle{attribute}=[draw=none]
\tikzstyle{defi}=[draw=gray,dashed]
\tikzstyle{textnode}=[draw=gray,fill=gray!30]

\input{introduction}

\input{technologies}

\input{translation}

\input{rules}

\input{evaluation}

\input{conclusion}

% anhänge
\appendix

\input{manuals}

\input{sources}

\backmatter			% abtrennung für verzeichnisse

% hier die verzeichnisse
%\listoffigures
\listoftables

% Bibliograhpy
\bibliographystyle{alphadin} 	% buchstaben + jahr und sortiert
\bibliography{Literature}

\clearpage
\erklaerung

\end{document}
